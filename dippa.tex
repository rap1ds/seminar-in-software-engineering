
\documentclass[conference]{IEEEtran}

\usepackage{cite}

\usepackage{verbatim} 

\ifCLASSINFOpdf
  % \usepackage[pdftex]{graphicx}
  % declare the path(s) where your graphic files are
  % \graphicspath{{../pdf/}{../jpeg/}}
  % and their extensions so you won't have to specify these with
  % every instance of \includegraphics
  % \DeclareGraphicsExtensions{.pdf,.jpeg,.png}
\else
  % or other class option (dvipsone, dvipdf, if not using dvips). graphicx
  % will default to the driver specified in the system graphics.cfg if no
  % driver is specified.
  % \usepackage[dvips]{graphicx}
  % declare the path(s) where your graphic files are
  % \graphicspath{{../eps/}}
  % and their extensions so you won't have to specify these with
  % every instance of \includegraphics
  % \DeclareGraphicsExtensions{.eps}
\fi


% *** ALIGNMENT PACKAGES ***
%
%\usepackage{array}
% Frank Mittelbach's and David Carlisle's array.sty patches and improves
% the standard LaTeX2e array and tabular environments to provide better
% appearance and additional user controls. As the default LaTeX2e table
% generation code is lacking to the point of almost being broken with
% respect to the quality of the end results, all users are strongly
% advised to use an enhanced (at the very least that provided by array.sty)
% set of table tools. array.sty is already installed on most systems. The
% latest version and documentation can be obtained at:
% http://www.ctan.org/tex-archive/macros/latex/required/tools/

% correct bad hyphenation here
\hyphenation{op-tical net-works semi-conduc-tor}


\begin{document}
%
% paper title
% can use linebreaks \\ within to get better formatting as desired
\title{Effective communication media for customer feedback in agile software projects}

% author names and affiliations
% use a multiple column layout for up to three different
% affiliations
\author{\IEEEauthorblockN{Mikko Koski}
\IEEEauthorblockA{Information networks, School of Science, Aalto University\\Espoo, Finland\\
Email: mikko.koski@aalto.fi}}

% make the title area
\maketitle


\begin{abstract}
%\boldmath

Agile software development methods are all about customer feedback. However, no research have been done about how efficiently communicate feedback from customer to developer. What are the communication media which result the highest communication performance for feedback communication?

In this paper a theoretical approach was chosen in order to answer the research question \textit{what are the most efficient communication media to give feedback?} Based on literature review two communication media theories were selected: Media Richness Theory and Media Synchronicity Theory. These theories were applied to feedback communication in agile software projects.

The results were against the common wisdom that face-to-face is always the most suitable communication media. Instead, the two theories suggest that "leaner" media, such as emails should result better communication perfomance. 

\end{abstract}

\IEEEpeerreviewmaketitle



\section{Introduction}
% no \IEEEPARstart
% This demo file is intended to serve as a ``starter file''
% for IEEE conference papers produced under \LaTeX\ using
% IEEEtran.cls version 1.7 and later.
% You must have at least 2 lines in the paragraph with the drop letter
% (should never be an issue)
% I wish you the best of success.

% This is how you use references \cite{IEEEhowto:IEEEtranpage}. Another example of reference usage: \cite{IEEEexample:tamethebeast}

% This is test

% \hfill mds
 
% \hfill January 11, 2007

Since the rise of agile software development methods, efficient customer communication have been taken seriously in software projects. The agile methods emphasize intense communication between customer and development organization instead of exchanging documents. Communication plays a big role in software projects of today. In previous research it has been shown that lack of communication and customer involvement is one of the biggest challenges faced by Agile teams \cite{2010hoda}. 

The Agile manifesto states in one of the twelve principles that "the most efficient and effective method of conveying information to and within a development team is face-to-face conversation." \cite{agilemanifesto} As a result the eXtreme Programming process demanded an onsite-customer \cite{2002wake}. However, this demand has later removed and replaced by a practice called Real Customer Involvement where the customer should be involved weekly \cite{2006korkala}.  Korkala et al. underline that because of the lack of Onsite-customer it is essential that communication and feedback mechanisms should receive special attention in agile development \cite{2006korkala}.

A lot of research has been conducted about communication in software projects but not very many with the focus on feedback communication. I believe that customer communication in software projects is concept that includes different situations requiring interactions between the customer and the development team. For example the communication required while doing planning is very different from the communication required while the customer is giving feedback. For example, Damian et al. have done research on using different communication media in requirements negotiation, but no such research have been done on customer feedback \cite{2000damian}. Thus, it makes sense to do research on this very specific communication sector.

Feedback is an important part of the communication because it enables customer to control the project and direct the development organization to the correct route. The lack of feedback or slow feedback cycle may result undesired outcomes.

This paper focuses on customer feedback in agile software project. The research methods are theoretical based on the existing literature and studies. At the end of the paper a new feedback tool Hannotaatio is going to be introduced. Hannotaatio is a visual website feedback tool which allows customer to give visual feedback to the development organization.

% Why is the customer feedback important?
% Various 

% Miksi Hannotaatio olisi hyvä työkalu?
% Miksi asia on edes tärkeää?
% Tärkeää koska ei ole on-site customeriä.

\section{Definition of customer feedback}

% Three phases of communication
% Before-, mid-, end-communication

Customer feedback is a part of communication between the customer and the development organization during the software project. To be able to give feedback the customer obviously has to have something (e.g. a design document or a piece of working software) from which she can give the feedback. Thus, the feedback communication can take place only after the development or design process has already started.

Feedback can be given about various aspects of the project. Feedback can be given about working methods, project management, communication practices or the actual output of the effort the development team has conducted. The output of the work effort can be e.g. technical design of the product, a visual design of the product or a piece of working software. This paper concentrates on feedback about \textit{what} the team has done, not \textit{how} the team has done the work. The main focus is on feedback about the working piece of software the team has delivered.

Feedback can be given from customer to developer or from developer to peer developer. This paper focuses on feedback from customer to developer. 

Since the context of the paper is in agile development, by customer I mean Product Owner. Product Owner is a agile team member who is responsible of the product and its success. Product Owner maintains the product backlog and  prioritizes the items in the backlog. This gives Product Owner an ability to control the course of the project. Because Product Owner is responsible of the project outcome, she is also a person who most likely provides the team with the most valuable feedback.

Product Owner does not have to be member of the customer organization. However, in this paper customer refers to Product Owner because she is ultimately the person who can answer the question "Is this the right thing to do for the customer?" \cite{2012sutherland}

\begin{comment}

- Asiakas TAI henkilö, jolla on kyky olla asiakkaan edustajana
- Ikkuna ulkomaailmaan
- Proxy asiakkaan suuntaan
- Käsitelläänkö tässä molempia? Palautetta Product ownerilta JA palautetta asiakkaalta

\end{comment}

In agile software development the software project is divided into iterations. In each iteration, various communication phases occur at various stages of the iteration. According to Bhalerao these communication phases consist of primary, mid and end phases \cite{2010bhalerao}.

Bhalerao states, that feedback involves in the end iteration. Primary phase communication deals with information gathering and happens in the early phase of the iteration. Mid-phase communication takes place after the initial task gathering and prioritization when the development team is implementing the iteration tasks. At this stage, technical communication is involved. \cite{2010bhalerao}

In this paper I am not considering feedback to be given only in the end phase of iteration. In agile context, teams are encouraged to utilize continuous integration principle \cite{2002wake}. Continuous integration allows the customer to always have access to the latest version of the software or at least once per day if nightly builds are used instead. Thus, it is most likely that customer will give feedback to the team also in the mid phase of the iteration. This is also encouraged to provide a rapid feedback cycle.

\begin{comment}

\section{Current communication methods for giving and receiving feedback}

The communication methods can be characterized based on the different media capabilities proposed by media richness theory and media synchronicity theory. In addition to these capabilities, the media either synchronous or asynchronous character and they or may not require sharing the same physical location.

In the following chapters various methods for giving and receiving feedback in current agile software projects are investigated.

\textbf{TODO:} Synchronous (face-to-face) vs. asynchonous methods (almost all the other)

\textbf{TODO:} Same physical location vs. different physical location

\subsection{Face-to-face}

Methods for giving feedback vary from document sharing to face-to-face conversation. In agile context tight customer collaboration is emphasised and the preferred communication method is face-to-face conversation. Extreme Programming (XP) even demanded an on-site customer to enable rapid feedback and high-bandwidth communication between customer and the development team \cite{2002wake}. % Wake page 1, lol

In Scrum, on-site customer is not mandatory (LÄHDE). When customer is off-site, the communication methods in the mid iteration falls back to email and phone, as Bhalerao states \cite{2010bhalerao}. The face-to-face feedback is given at the end of the iteration in the iteration demo session.

Bhalerao states that feedback communication should be done face-to-face \cite{2010bhalerao}. Also Korkala et. al. underlines the importance of face-to-face communication by arguing that it should be the default communication method in agile development \cite{2006korkala}.

Despite the advantages of face-to-face communication it is also a costly communication method. Face-to-face communication is synchronous method which requires shared physical location. In software projects where customer is not on-site, other feedback methods than have to be exercised.

% Käytä tätä KORKALA: 
% Since the customer is not any longer required for fulltime presence, the already important roles of communication and feedback become even more essential. Thus it is essential that the communication and feedback mechanisms should receive special attention in agile development.

\subsection{Video conference and telephone}

\subsection{Email}

\subsection{Chat rooms}

\textbf{TODO: Synchronous vs. asynchonous methods, Email, Video conference, documents etc.}

\section{Communication and feedback methods in software projects}

\end{comment}

\section{Research methods}

This paper investigates the communication between customer and the development organization having a focus on the feedback communication. The context is in agile software development. Strong emphasis is put in inspecting the different methods of feedback communication and investigating how these methods can be used effectively. The research question is: \textit{What are the most efficient communication media to give feedback?}

The research methods used in this paper are theoretical based on existing theories and studies. A literature review was conducted around the subject of communication methods and media use.

I selected two media communication theories in order to answer the research question. The two theories were Media Richness Theory (MRT) and Media Synchronicity Theory (MST). These two theories describe communication media capabilities and suitability for various situation, thus providing a good framework to answer what are the efficient communication media in case of feedback communication.

The media richness theory was chosen because it is the most well-known and widely used theory about communication media \cite{1986daft}\cite{2006korkala}. However, media richness theory has faced some criticism and thus new theories have been created to fill the gaps in the media richness theory \cite{1999dennis}. The theory was introduced before the era of new media (email, videoconferences etc.) so I decided to include also another communication media theory in the research.

The second theory chosen was media synchronicity theory which is derived from media richness theory. This theory was chosen because it is suitable to new communication media, which are in important position in today's communication media in software industry. The theory also describes in more detail the different capabilities of each communication media thus making it easier to evaluate capabilities of different media.

Because of the criticism towards the media richness theory and because of the better applicability of media synchronicity theory to new media the emphasis in this paper is on media synchronicity theory.

\subsection{Media richness theory}

Daft et. al. have proposed a theory of media richness \cite{1986daft}. Media richness theory (MRT) is well-known and widely used even though it has been criticized \cite{2006korkala} \cite{1999dennis}. 

According to MRT different communication methods can be ranked based on their "richness". Daft et al. define the richness as the ability of information to change understanding within a time interval. The richness is derived from the capacity of immediate feedback, number of cues and channels utilized, personalization and language variety. Daft et al. list the following media classifications in order of decreasing richness: face-to-face communication, telephone, personal documents, impersonal written documents and numeric documents \cite{1986daft}. To this list Korkala et. al. have added videoconference and email among others \cite{2006korkala}. The richness of videoconference is between face-to-face and telephone while the richness of emails goes to same category as personal documents.

Media richness theory utilizes concepts of uncertainty and equivocality. \textbf{Uncertainty} exists if information can be interpreted unambiguously but there is a lack of information. \textbf{Equivocality} exists when there are multiple and possibly conflicting interpretations although the amount of information is sufficient. As equivocality rises a greater amount of negotiations is required to reach a consensus on one interpretation. \cite{1999dennis}

MRT argues that certain communication media are more suitable for certain tasks. A richer media is preferred for high equivocal task while leaner media are suitable for tasks with low equivocality. 

\subsection{Media synchronicity theory}

Dennis and Valacich have criticized MRT for various reasons. First, empirical researches of media richness theory have not been convincing \cite{1998dennis} \cite{1997elshinnawy}. Second, empirical studies have shown that the media actually used for different communication tasks do not match with the Media Richness Theory. Korkala et. al. achieved the same result as they noticed email was commonly used communication practise even though "richer" communication methods were encouraged \cite{2006korkala}. Third, Dennis and Valacich argue that in contrast to MRT, one cannot rank communication methods "rich" and "poor". \cite{1999dennis}

In the theory of Media Synchronocity, Dennis and Valacich formed a list of five media characteristics that can affect communication. The characteristics are transmission velocity (also known as immediacy of feedback \cite{1999dennis}), parallelism, symbol sets, rehearsability and reprocessability. They evaluated various communication methods from face-to-face discussions to written documents based on the five characteristics. The result of the evaluation was that methods can not be ranked from "best" to "worst", or from "richest" to "poorest" as Media Richness Theory argues \cite{2008dennis}. 

Media synchronocity theory states that communication is composed of two primary processes: conveyance and convergence. \textbf{Conveyance} process exists when there is a need for transmission of a new information. The information receiver processed the new information and creates a mental model of the situation. \textbf{Covergence} process means discussing of the information processed by the individuals. The target is to change views of each interpretation and agree on mutual understanding. The individuals' familiarity with each other, the task they are performing and the communication media they are using affects to the relative amount of these two processes. For familiar communication context, the emphasis is on the conveyance process \cite{2008dennis}.

\section{Results}

\subsection{The nature of feedback communication}

Before applying the MRT and MST theories the nature of customer feedback communication has to be defined. In customer feedback communication, customer is the source of information. In most cases the amount of information available from the customer is sufficient for the team to execute follow up actions. However, in some cases the customer may be unsatisfied but unable to provide necessary feedback for the team to come up with an appropriate solution. It can be stated that the level of information available varies. In terms of media richness theory this means that the level of uncertainty varies but in general it can be stated that the level of uncertainty is medium. 

The development team has to interpret the feedback from customer after receiving it. The feedback from the customer can be ambiguous even if the amount of information available is sufficient. When a developer interprets the feedback received from customer multiple questions may arise: Are we talking about the same part of the software? Why is this a problem in the first place? How it should be fixed? Because of the ambiguous nature of feedback and possibility of conflicting interpretations, in the context of media richness theory this means that feedback communication is affected by equivocality. However, since the customer and the team are having feedback communication around familiar and known subject (the software product) it can be argued that the level of equivocality is not the highest one, instead, medium.

As stated already, feedback communication is held in a context which is familiar to the individuals. The individuals are most likely used to work with each other and they are familiar with the tasks they are working on and the media they are using for communication. According to media synchronicity theory, in a familiar communication context the emphasis on the communication should be on conveyance process. The theory states that conveyance processes are best served by media with capabilities supporting low synchronicity.

\subsection{Feedback communication according to MRT}

The media richness theory proposes that "richer" communication media are more suitable for tasks with high equivocality where as "leaner" media are more suitable for tasks with low equivocality but high uncertainty. \cite{1999dennis} In the case of feedback communication we fall in the middle.

\begin{comment}
\textbf{TODO: Onko palaute yksiselitteisesti equivocality? MST:n mukaan ollaan hieman toisilla jäljillä}

\textbf{TODO: Lue MRT uudestaan ja koita ymmärtää onko just näin. Lue myös Korkalaa ja Bhaleraoa, jotka ovat käyttäneet ko. teoriaa}

In iterative software development various communication phases occur in an iteration. According to Bhalerao these phases are primary, mid- and end-iteration phases. \cite{2010bhalerao}

The primary phases consist of planning tasks. Before the feature implementation, the development team and the customer have to form a shared understanding of what will be implemented. As the consensus is formed it is agreed or documented as a specification for the implementation. In this phase high uncertainty and high equivocality exists. 

In the mid-iteration phase the implementation of the agreed features for the iteration has started. Mid-iteration communication consist of corrective questions regarding the specification. Uncertainty is high while equivocality is low since the high-level consensus has been formed in the previous phase. As the development team starts delivering implemented features feedback communication takes place.

The end-iteration communication consist of feedback communication. The primary task in this phase is to validate the implemtented features. The main method for validation is an iteration demo and customer feedback.

The feedback the customer gives can be unambigious. When a developer interprets the feedback received from customer multiple questions may arise: Are we talking about the same part of the software? Why is this a problem in the first place? How it should be fixed?

Since the feedback can be unambigous and difficult to interpret it can be argued that high equivocality involves in feedback communication. According to media richness theory rich communication methods should be used in feedback communication. Also Bhalerao suggegests the same by arguing that face-to-face communication is the preferred mode for feedback communication for instant feedback \cite{2010bhalerao}.

\end{comment}

\subsection{Feedback communication according to MST}

As stated earlier, media with capabilities supporting low synchronicity best serves feedback communication. According to media synchronicity theory, the five media capabilities have different capabilities to support synchronicity. Evaluating these capabilities in the context of feedback it can be seen what kind of capabilities an effective feedback method has.

\textbf{Transmission velocity} is the speed at which medium is capable of transmitting the message to recipient. From feedback point-of-view, transmission velocity is important but not as important as it is for e.g. planning tasks. For example, communication context for planning is novel where as feedback is given in a context where feedback sender and receiver are familiar with the subject. When the context is familiar conveyance should be emphasized. To support conveyance a communication method with lower synchronicity level should result in better communication performance. High transmission velocity supports synchronicity, thus in conclusion, for feedback purposes where conveyance process is emphasized, communication method with lower transmission velocity should be used according to Dennis et al. \cite{2008dennis}

\textbf{Parallelism} describes the medium's capability for multiple parallel communication sessions \cite{2008dennis}. Synchronous communication methods such as face-to-face communication and telephone support poorly parallelism where as asynchronous computer-aided methods such as chat rooms and emails support parallelism well. Parallelism has negative impact on media synchronicity. Thus, media with high parallelism should be used for feedback purposes. \cite{2008dennis}

\textbf{Symbol set} describes the number of ways in which a medium allows information to be encoded for communication \cite{2008dennis}. For example face-to-face communication has a higher number of symbol sets than written document since face-to-face communication can transfer also vocal tones and physical gestures. Symbol sets can be natural (physical, visual, verbal) or less natural (written or typed). More natural symbol sets support higher synchronicity, however, using a medium with a symbol set better suited to the content of message will improve information transmission and processing \cite{2008dennis}. For feedback this means that a verbal description of a activity on a Web site can be less effective than a visual demonstration and a verbal description or a series of annotated screen shots with a written description \cite{2008dennis}.

\textbf{Rehearsability} stands for the ability to fine tune the message before sending it. High rehearsability reduces possibilities to misunderstandings as the sender can carefully fine tune the message to describe exactly what she means to. However, rehearsability adds delays to the conversation. \cite{2008dennis} 

From the viewpoint of feedback, rehearsability is important. An ill-advised comment from customer about an implemented feature may give a wrong impression to the developer who may end up doing a change that the customer did not actually intended from the first place. In addition, giving a negative feedback to the development team in an indiscreet way may reduce developers' motivation.

\textbf{Reprocessability} describes the possibility to reprocess the transmitted message. The ability to reread the message increases the understanding of the content, but adds delays to the conversation. \cite{2008dennis}

The understanding of the feedback given by the customer increases if the developer can reprocess the feedback. This is especially true if the message is communication via a medium which does not support symbol set suited to the content of the message. 

In many occasions the received feedback requires for actions. The required action may not be executed immediately. If for example a developer makes a change based on the customer feedback after a couple of days of receiving the feedback, the reprocessability plays a great role.

\begin{table}[!h]
% increase table row spacing, adjust to taste
\renewcommand{\arraystretch}{1.3}
% if using array.sty, it might be a good idea to tweak the value of
% \extrarowheight as needed to properly center the text within the cells
\caption{Media capabilities and their importance for feedback}
\label{table:capabilities}
\centering
% Some packages, such as MDW tools, offer better commands for making tables
% than the plain LaTeX2e tabular which is used here.
\begin{tabular}{|p{1.8cm}|p{4.2cm}|p{1.3cm}|}
\hline
\textbf{Media \newline capability} & \textbf{Description} & \textbf{Importance for \newline feedback}\\
\hline
Transmission \newline velocity & The speed in which the information is transported from individual to another & Low\\
\hline
Parallelism & Capability for multiple parallel communication sessions & High\\
\hline
Natural symbol set & Diversity of symbols which allows information encoding. Natural symbols are vocal tones and physical gestures etc. & Low\\
\hline
Rehearsability & The ability to fine tune the message before sending it & High\\
\hline
Reprocessability & The possibility to reprocess the transmitted message & High\\
\hline
\end{tabular}
\end{table}

In conclusion, feedback is given in a context, which is familiar to the individuals working with each other, thus moving the emphasis from convergence process to conveyance process. According to MST conveyance processes are best served by media with capabilities supporting low synchronicity. Media with low synchronicity are for example documents, fax, voice mail, asynchronous electronic mail (email) and asynchronous electronic conferencing \cite{2008dennis}. According to the results the capabilities of the most suitable feedback communication media are low transmission velocity, high parallelism, high rehearsability and high reprocessability. These results are listed in the table~\ref{table:capabilities}.

\begin{comment}

NOPEA PÄÄTELMÄ: 
MST: Feedback: 

Contexti: Familiar, käytännössä siis: high conveyance, low convergence ==> less synchronicity

P1: Lower synchronicity
P2: Trasmission velocity improves synchronicity -> Lower transmission velovity
P3: Parallelism lower synchronicity -> High parallelism
P4(a): Natural symbols support synchronicity -> Use unnatural symbols
P4(b): Symbols better suited to content support synchronicity -> Use ill suited symbols
P5: Rehearsability lowers synchronicity -> High rehearsability
P6: Reproducability lowers synch -> High reproducability

--> 

lower transmission velocity, high parallelism, reharsability is important, high reprocessability -> Elektoriseen muotoon

% Eri metodien vaikutus? Saako lämpimämpi metodi vauhditettua korjausta

\section{Call for new methods for feedback communication}

\textbf{TODO:} Why there's a need for new methods?

\textbf{TODO:} Introduce Hannotaatio

\textbf{TODO:} Evaluate Hannotaatio based on MRT and MST

\end{comment}

\section{Discussion and conclusion}

According to both media richness theory and media synchronicity theory, face-to-face communication is not the most efficient feedback communication method. Instead, "leaner" methods or methods with lower support for synchronicity should be used.

The result is in contrast to some previous researches and what the agile manifesto proposes. One of the twelve principles of the agile manifesto states that "the most efficient and effective method of conveying information to and within a development team is face-to-face conversation." \cite{agilemanifesto} In addition, Bhalerao states that at the end of the iteration when customer feedback is given, active communication methods should be utilized. By active communication methods Bhalerao means face-to-face communication or telephone communication \cite{2010bhalerao}.

However, despite the result being in contrast with some previous researches and the agile manifesto, it is not totally unexpected result. Even though face-to-face communication is effective for most communication tasks, it comes with a high price. The problems of face-to-face communication are well known one of the most significant being the need for shared time and physical location. This is clearly not always possible, for example for globally distributed projects.

Korkala et al. noticed in their research that the customers of the projects selected email as the main communication channel despite the fact that they have been encouraged to use "richer" communication media \cite{2006korkala}. One reason for this could be that the customers subconsciously selected a communication media, which in fact had the best cost-value ratio.

It has to be noticed that the result of this paper is a result of simplification. In the paper the customer feedback was considered as a day-to-day communication practice of a rather simple and straightforward well-known subject. This assumption was made because it can be assumed that the customer and development team are already familiar with the subject before the feedback is given since they have in the beginning of the iteration planned it together. However, in the real life situations can be more complicated. If the customer for example has not been able to participate to the planning of the feature the feedback is given from it may be the case that the customer is seeing the feature for the first time. In this kind of situation the communication context is a lot more complex.

The results of this paper were derived from theoretcial framework. In real file, communication between individuals is affected by multiple complex factors such as emotions and relationships between individuals. These factors definitely affect to the performance of the communication, but they were not taken into account by the theories used in this paper. Further research, especially empirical research, should be conducted about feedback communication performance in real software projects.

As proposed in the media synchronicity theory the best communication performance can be reached by combining different communication practices \cite{2008dennis}. For example, an email with a screenshot attached results a much better communication performance that the plain email. On the other hand the communication methods used today have not developed lately very much. We are still using the same communication channels as ten years ago, emails, chats, telephone, videoconferences etc. In addition, there are not many communication methods designed especially for feedback purposes. There is clearly need for new kind methods and practices for giving feedback.

One interesting feedback tool implemented by Aalto University students sponsored by Futurice is Hannotaatio \cite{hannotaatio}, which is a visual website feedback tool allowing customer to give feedback by drawing on top of the website under development. However, further research would be needed to draw conclusions about the effectiveness of Hannotaatio.

% An example of a double column floating figure using two subfigures.
% (The subfig.sty package must be loaded for this to work.)
% The subfigure \label commands are set within each subfloat command, the
% \label for the overall figure must come after \caption.
% \hfil must be used as a separator to get equal spacing.
% The subfigure.sty package works much the same way, except \subfigure is
% used instead of \subfloat.
%
%\begin{figure*}[!t]
%\centerline{\subfloat[Case I]\includegraphics[width=2.5in]{subfigcase1}%
%\label{fig_first_case}}
%\hfil
%\subfloat[Case II]{\includegraphics[width=2.5in]{subfigcase2}%
%\label{fig_second_case}}}
%\caption{Simulation results}
%\label{fig_sim}
%\end{figure*}
%
% Note that often IEEE papers with subfigures do not employ subfigure
% captions (using the optional argument to \subfloat), but instead will
% reference/describe all of them (a), (b), etc., within the main caption.


% An example of a floating table. Note that, for IEEE style tables, the 
% \caption command should come BEFORE the table. Table text will default to
% \footnotesize as IEEE normally uses this smaller font for tables.
% The \label must come after \caption as always.
%
\begin{comment}
\begin{table}[!t]
% increase table row spacing, adjust to taste
\renewcommand{\arraystretch}{1.3}
% if using array.sty, it might be a good idea to tweak the value of
% \extrarowheight as needed to properly center the text within the cells
\caption{An Example of a Table}
\label{table_example}
\centering
% Some packages, such as MDW tools, offer better commands for making tables
% than the plain LaTeX2e tabular which is used here.
\begin{tabular}{|c|c|}
\hline
Media capability & Importance for feedback\\
\hline
 & Four\\
\hline
\end{tabular}
\end{table}
\end{comment}

% Note that IEEE does not put floats in the very first column - or typically
% anywhere on the first page for that matter. Also, in-text middle ("here")
% positioning is not used. Most IEEE journals/conferences use top floats
% exclusively. Note that, LaTeX2e, unlike IEEE journals/conferences, places
% footnotes above bottom floats. This can be corrected via the \fnbelowfloat
% command of the stfloats package.



% conference papers do not normally have an appendix

% use section* for acknowledgement


% trigger a \newpage just before the given reference
% number - used to balance the columns on the last page
% adjust value as needed - may need to be readjusted if
% the document is modified later
%\IEEEtriggeratref{8}
% The "triggered" command can be changed if desired:
%\IEEEtriggercmd{\enlargethispage{-5in}}

% references section

% can use a bibliography generated by BibTeX as a .bbl file
% BibTeX documentation can be easily obtained at:
% http://www.ctan.org/tex-archive/biblio/bibtex/contrib/doc/
% The IEEEtran BibTeX style support page is at:
% http://www.michaelshell.org/tex/ieeetran/bibtex/
\bibliographystyle{IEEEtran}
% argument is your BibTeX string definitions and bibliography database(s)
\bibliography{IEEEabrv,ref}
%
% <OR> manually copy in the resultant .bbl file
% set second argument of \begin to the number of references
% (used to reserve space for the reference number labels box)
% \begin{thebibliography}{1}

% \bibitem{IEEEhowto:kopka}
% H.~Kopka and P.~W. Daly, \emph{A Guide to \LaTeX}, 3rd~ed.\hskip 1em plus
%   0.5em minus 0.4em\relax Harlow, England: Addison-Wesley, 1999.

% \end{thebibliography}




% that's all folks
\end{document}


